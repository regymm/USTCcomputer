% -*- coding: UTF-8 -*-
\documentclass[UTF-8,cs4size]{ctexart}


\title{电磁学小论文 —— 带电粒子在电磁场中运动的计算机模拟程序的简单实现}
\author{PB17000002  古宜民}
\date{2018年6月}

%\bibliographstyle{plain}


\begin{document} \normalsize


\maketitle
\begin{center}
	摘要
\end{center}

带电粒子在电磁场中运动是电磁学的重要组成部分,是电磁学的理论与实验基础,既可以使用已知场来研究带电粒子的性质,如质谱仪;也可以用已知粒子来考察场的性质,无论在科学研究还是应用实践领域都有必不可少的作用。但真正进行实验测试需要专业设备和人员,是一件成本较高的事情,而计算机模拟不但方便快捷,还能得到更加准确或抽象的模型,有重大意义。本文以静电磁场基本方程:库伦定律、叠加原理和毕奥——萨伐尔定律为基础,不考虑相对论效应等其他因素,使用Python语言对带电粒子在静电磁场中运动轨迹进行模拟并可视化,即实现一个简易的物理引擎,并使用实际物理模型进行验证。自己从零开始分析问题、实现程序,对物理理解和编程能力都有很大锻炼。


本文将介绍模拟算法、程序设计结构和实现过程,并展示可视化结果,最后分析程序结果和可改进方面。
\clearpage
\section{电磁场基本方程}
\subsection{基本概念及定理}
\subsubsection{静电场}
电场由电荷产生,定义式为:
\begin{equation}
	\sum i_{in} = \sum i_{out}
\end{equation}
两电荷间作用力满足库仑定律:
\begin{equation}
	\sum i_{in} = \sum i_{out}
\end{equation}
其成立条件为静止(或低速)电荷。


实验证明两静止点电荷间作用力不因第三个点电荷的存在而改变,即满足叠加原理:
\begin{equation}
	\sum i_{in} = \sum i_{out}
\end{equation}
由此可求得空间任一点处电场强度。
\subsubsection{静磁场}
磁场由电流产生,其值由毕奥——萨伐尔定律给出:
\begin{equation}
	\sum i_{in} = \sum i_{out}
\end{equation}
粒子在磁场中受力与粒子运动情况有关,有:
\begin{equation}
	\sum i_{in} = \sum i_{out}
\end{equation}
由此可求得空间任一点处磁场强度及粒子在磁场中受力。
而历史上还有一种用电场成因类比磁场的理论,即磁荷理论,该理论认为磁场与电场类似,由“磁荷”产生。该理论虽然被证明是不正确的,但在一些模拟任务中可以简化模型并获得直观的、较好的效果。
%\subsubsection{元件}
%\subsubsection{电路分析常用名词}
%\paragraph{支路} 电路中一个或若干个二段元件依次串联且流经的是同一个电流的电路分支称为一个支路。
%\paragraph{结点} 电路中三个及以上支路的连接点,称为结点。
%\paragraph{回路} 电路中由若干个支路组成的闭合路径,称为回路。
%\paragraph{网孔} 平面电路中不含支路的回路,称为网孔。
%\subsubsection{基尔霍夫定律}
%基尔霍夫定律是电路理论的基石,对电路分析与求解有重大意义。
%\paragraph{\textbf{基尔霍夫电流定律(Kirchhoff's Current Law, KCL)}} 电路中的任一结点,在任意时刻,流入该结点的全部支路电流之和等于流出该结点的全部支路电流之和,即
%\begin{equation}
	%\sum i_{in} = \sum i_{out}
%\end{equation}
%\paragraph{\textbf{基尔霍夫电压定律(Kirchhoff's Voltage Law, KVL)}} 电路中任一回路,在任意时刻,沿该回路闭合路径绕行一周的全部支路电压的代数和为零,即
%\begin{equation}
	%\sum_{k=1}^{b}u_k = 0
%\end{equation}
%其中$b$为回路中包含支路个数,$u_k$为回路中第$k$个支路电压。
%由图论知识可以证明,对于具有$n$个结点的电路,其独立的KCL方程有$n-1$个,通常取一个结点为参考结点,其余为独立结点;对于具有$n$个结点,$b$条支路的电路,其独立的KVL方程有$b-n+1$个。
%\subsection{计算机方法}
%计算机电路分析的方法和思路与平时计算与解题不同,一般解题看中计算的简便性与技巧性,但计算机分析则应看中方法的通用性,即一种模型能解一整类问题,而计算的繁琐则不太重要了。

\section{程序结构及计算算法}
\subsection{计算方法}
\subsubsection{欧拉法}
\subsubsection{龙格——库塔法}
\subsection{程序结构与设计细节}
\subsubsection{程序结构}
\subsubsection{可视化}
\subsubsection{设计细节}
\section{物理模型的验证、分析及可视化结果}
\subsection{模型一:点电荷电场中的匀速圆周运动}
\subsection{模型二:匀强磁场中的等距螺旋线运动}
\subsection{模型三:地磁场束缚带电粒子(磁镜效应)}
\section{模型的不足之处与可能的改进方案}
\subsection{计算方法的改进}
\subsection{计算效率的优化}
\subsection{可视化的改进}
\end{document}









